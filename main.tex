\documentclass[10pt, conference]{IEEEtran}
% \IEEEoverridecommandlockouts
% The preceding line is only needed to identify funding in the first footnote. If that is unneeded, please comment it out.
\usepackage{cite}
\usepackage{hyperref}
\usepackage{amsmath,amssymb,amsfonts}
\usepackage{algorithmic}
\usepackage{graphicx}
\usepackage{textcomp}
\usepackage{xcolor}
\usepackage{orcidlink}
\usepackage{acronym}
\def\BibTeX{{\rm B\kern-.05em{\sc i\kern-.025em b}\kern-.08em
    T\kern-.1667em\lower.7ex\hbox{E}\kern-.125emX}}

\input{preamble}
% !TEX root = main.tex

% \acrodef{APK}{Android Application Package}
% \acrodef{AI}{Artificial Intelligence}
% \acrodef{WCAG}{Web Content Accessibility Guidelines}
% \acrodef{UI}{User Interface}
% \acrodef{AOM}{Accessibility Object Model}
% % \acrodef{a11y}{Accessibility}

\acrodef{VR}{virtual reality}
\acrodef{NLP}{natural language processing}
\acrodef{HCI}{human-computer interaction}
\acrodef{AAC}{augmentative and alternative communication}
\begin{document}

\title{EyeNav: Accessible Interaction and Testing using Eye-tracking and Natural Language Processing}

\author{
    \IEEEauthorblockN{
        Juan Diego Yepes-Parra \orcidlink{0009-0007-0672-1473}, 
        Camilo Escobar-Velásquez \orcidlink{0000-0001-8414-9301}
    }
    \IEEEauthorblockA{
        \textit{Universidad de los Andes, Colombia} \\
        \{j.yepes, ca.escobar2434\}@uniandes.edu.co
    }
}

\maketitle

\begin{abstract}
% TODO 
%! tengo que citarme a mi mismo??
% use acronyms
% github
% screencast
%   - Briefly state: envisioned users; SE challenge addressed; key features of the tool;
%     methodology integration (eye-tracking + NLP); validation status (studies done or planned).
%   - Append:
%     * YouTube link to screencast
%     * URL for open-source code & datasets
In the field of \ac{HCI}, alternative interaction methods are becoming increasingly popular and commercialy available. From this opportunity came EyeNav, a system that combines eye-tracking and natural language processing (NLP) to enhance accessibility and enable automated test generation. The integration of these technologies for intuitive web interaction, using pointer control via gaze and natural language processing for interpreting user intentions, also presents a record-and-replay module for generating automated test scripts. Envisioned for users with motor disabilities, developers, usability testers, and general users interested in exploring novel multimodal web interactions. %add the other modalities
The ultimate goal is to demonstrate that this tool can be used not only as a possible assistive technology but also as an innovative approach to software testing. The tool is available publicly \href{https://tinyurl.com/eyenav-repo}{https://tinyurl.com/eyenav-repo}, with a walkthrough video \href{https://tinyurl.com/eyenav-repo}{https://tinyurl.com/eyenav-repo} %! cambiar uno de estos por el video
\end{abstract}


\begin{IEEEkeywords}
% TODO
Eye-tracking; Automated Test Generation; Assistive Technology; Natural Language Processing; Web Applications; Accessibility.
\end{IEEEkeywords}

% must include:
% the envisioned users;
% the software engineering challenge the tool addresses;
% the methodology it implies for its users;
% the results of validation studies already conducted (for mature tools) or - the design of planned studies (for early prototypes).

% A demonstration submission must not exceed four pages (including all text, references, and figures);
% Authors are required to submit a screencast of the tool, with the video link attached to the end of the abstract;
% Authors are encouraged to make their code and datasets open source, with the link for the code and datasets attached to the end of the abstract;



% !TEX root = main.tex

\section{Introduction}

Alternative interaction methods are increasingly common in modern computing systems. Devices like smartphones, wearables~\cite{tobii_glasses_x}, and headsets~\cite{apple_vision_pro_2025,playstation_vr2_specs,vive_pro2_2025} incorporate novel input technologies that enable more natural, inclusive, and adaptive user experiences~\cite{dondi2023gazehci, fernandes2023eyevr}. These innovations are especially valuable for accessibility, offering alternative ways for users with physical limitations to engage with digital content~\cite{hsieh2024increasing}.

While eye-tracking has long been studied, its integration into consumer devices is making it a practical input method, enabling precise, intuitive, and hands-free interaction~\cite{huang2024visionpro}. In contrast, NLP is already a well-established modality, powering spoken interfaces from voice-enabled coding tools~\cite{serenade2025} to smart assistants. However, these systems still face challenges in accurately interpreting user intent, underscoring the need for robust semantic and contextual processing~\cite{mozafari2020chatbot, liu2024chatgpt}.

EyeNav introduces a novel multimodal input approach by combining real-time eye-tracking and natural language commands for web interaction. Implemented as a Chrome extension, it supports gaze-based navigation, voice commands, and automated testing via a Gherkin-based record-and-replay module. Originally designed for users with motor impairments, EyeNav also benefits developers seeking hands-free browser control, usability professionals conducting accessibility evaluations, and users exploring multimodal interaction. By merging assistive input with test automation, EyeNav advances the integration of accessible technologies into mainstream web environments.%
% !TEX root = main.tex

\section{Related Work}

\subsection{Eye-tracking in HCI}

Eye-tracking has long been used to study human cognition and behavior, especially in behavioral and psychological research. Early applications focused on observation rather than interaction. For example, Zelinskyi et al. developed a Chrome extension for collecting eye-tracking data for behavioral analysis~\cite{zelinskyi2024eyetracking}, and Jacob and Karn emphasized its value in usability research for evaluating user interfaces~\cite{jacob2003commentary}.

As the field of \ac{HCI} evolved, researchers began exploring eye-tracking as an input method. Gips et al. introduced an early eye-controlled system to support users with motor impairments~\cite{gips1996eagleeyes}, and more recent work has expanded into areas like immersive experiences~\cite{dondi2023gazehci} and AI-powered image editing~\cite{karlander2023ai}. Modern AR/VR devices such as the Meta Quest Pro, Pico 4 Pro, and Apple Vision Pro now integrate eye-tracking natively, with the latter relying entirely on eye and hand gestures for interaction~\cite{huang2024visionpro}.

Eye-tracking also plays an increasing role in accessible technologies. Wang et al.\ developed GazePrompt, a reading aid for low-vision users that responds to gaze-based behavior by offering visual and auditory assistance~\cite{wang2024gazeprompt}. Similarly, commercial devices like the Tobii Dynavox TD Pilot allow users to control iPad-based AAC systems using only their eyes~\cite{poster2025td}.

\subsection{Speech recognition NLP in HCI}

\ac{NLP} has similarly demonstrated significant potential in assistive contexts within HCI as an input method~\cite{song2024review}. 

Conversational assistants represent a widely adopted interface paradigm enabled by NLP. Established systems such as Siri~\cite{apple_siri}, Google Assistant~\cite{google_assistant}, and Alexa~\cite{amazon_alexa} have become integrated deeply to many consumer devices. More recently, generative AI has facilitated the emergence of other assistants, including ChatGPT~\cite{openai_chatgpt} and Gemini~\cite{google_gemini}. Notably, these assistants typically operate as discrete applications rather than as embedded control layers, meaning users must actively invoke them within specific contexts rather than relying on them for continuous, system-wide interaction.

In the context of accessibility, Girón-Bastidas et al.\ emphasize the effectiveness of NLP-based technologies in supporting users with hearing impairments, highlighting their utility in communication and interface adaptation~\cite{gironbastidas2019nlp}. Martínez et al. also developed a tool for simplifying online content, enabling understandability for people with cognitive disabilities~\cite{martinez2024tool}. Avalos et al. propose a context-based model that allows for browsing the web through voice. The system utilizes user utterances to command the system entirely; thereby enabling users with motor disabilities to engage with web content~\cite{avalos2025context}. These efforts exemplify how NLP can be highly adaptable to many use-cases, and types of disabilities.

NLP also finds application across a variety of other domains. In educational contexts for instance, NLP-enabled assistive technologies have been shown to enhance learning by supporting more immersive and interactive experiences~\cite{terzopoulos2020voice}. Additionally, conversational agents have been employed to facilitate a wide range of instructional and communicative interactions~\cite{liu2024chatgpt}.

\subsection{Multimodal interfaces}

The integration of gaze data with speech input represents a growing area of research in multimodal HCI. Khan et al.\ propose a system that combines eye-tracking with voice commands for implicit interaction, using gaze to reinforce the user's spoken intent~\cite{khan2022integrating}. Lee et al.\ introduce GazePointAR, a wearable system that leverages both eye-tracking and speech recognition to support real-time image recognition and context-aware assistance~\cite{lee2024gazepointar}. Similarly, Zhao et al.\ present EyeSayCorrect, an autocorrection system that uses both modalities to improve speech-based text input~\cite{zhao2022eyesaycorrect}.

\subsection{The Future of the Web}

The increasing prevalence of eye-tracking in mainstream consumer devices presents new challenges and opportunities for web design. Panwar examines the evolution of web applications from their static origins to highly dynamic, complex systems. The study emphasizes the growing importance of multimodal interaction, predicting that future interfaces will increasingly combine touch, voice, text, and gesture to meet user expectations~\cite{panwar2024webevolution}. This underlines the need for web interfaces to adapt for supporting eye-based interaction, favoring larger and more visually distinct elements over traditional hyperlink-based controls~\cite{apple2024spatialweb}.


\subsection{Record-and-Replay Testing}

Record-and-replay testing enables developers to capture user interactions during runtime and replay them for debugging, regression testing, or usability evaluation\cite{vasquez2018continuous, moran2016automatically}. In this context, this approach provides a practical means of reproducing complex interaction sequences in controlled environments. Consequently, record-and-replay testing is becoming a vital component of usability studies and system validation in immersive and gaze-aware interfaces.



% !TEX root = main.tex

\section{EyeNav}

\begin{figure}[h]
	\centering
	\vspace{-10pt}
	\includegraphics[width=0.7\linewidth]{imgs/diagram-context.png}
	\caption{Context diagram of the system.}
	% \vspace{-10pt}
	\label{fig:context}
\end{figure}

This section outlines the EyeNav system based on its workflow, as illustrated in Fig.~\ref{fig:context}. EyeNav is composed of 2 main modules: (i) a Chrome Extension sidebar and (ii) a backend module built in python. The Chrome extension sidebar, which functions as an adjacent webpage alongside the main content, presents the available verbal commands, allows to initiate a interaction session, and shows the interpreted verbal commans in realtime once a session has started (See Fig.~\ref{fig:requirements}). Once a session begins, the system orchestrates multiple components, including voice recognition, eye tracking, and interaction logging.

\begin{figure}[h]
	\centering
	\includegraphics[width=180pt]{imgs/system-requirements.png}
	\caption{A graphic of what the system looks like}
	% \vspace{-13pt}
	\label{fig:requirements}
\end{figure}

User input is captured via an eye-tracker and microphone, while the underlying processing occurs in a backend service rather than on the frontend. Concurrently, user interactions, such as clicks, text input, and navigation events, are recorded by a logging module. These events are compiled into an executable test file, which can later be replayed using Gherkin-based test execution tools like Selenium\cite{garcia2024selenium} or Kraken\cite{ravelo2023kraken}.


\subsection{High-Level Architecture}

\begin{figure}
    \centering
    \includegraphics[width=200pt]{imgs/components-diagram.jpg}
    \caption{Components diagram}
    \vspace{-13pt}
    \label{fig:components-diagram}
\end{figure}

Figure ~\ref{fig:components-diagram} presents the complete component architecture of the system. The Chrome extension serves as the front-end interface, providing real-time textual feedback for recognized voice commands and capturing user interaction events. Click events are detected on the frontend and forwarded as HTTP POST requests to the backend to preserve contextual information, such as the associated HTML tag. 

Eye tracking is powered by the Tobii Pro Nano, a single-camera dark/bright pupil system with corneal reflection and a typical latency of approximately 17 ms~\cite{tobiiabndpronano}. The gaze-driven pointer control module uses the \verb|tobii-research| Python SDK to access real-time gaze data and interpolate cursor movement accordingly.

Voice commands are captured through a microphone and transcribed using the Vosk speech recognition engine\cite{vosk2025}, which operates entirely on-device for privacy and low latency. Recognized phrases are matched to predefined command templates and sent to the backend over WebSocket connections, enabling minimal response delay.

The backend integrates data from both the Tobii eye-tracker and the Vosk engine to interpret user intent, translate inputs into executable UI commands, and log all interactions. These are executed using Kraken, a behavior-driven testing framework built on WebdriverIO\cite{webdriverio}, supporting usability and regression testing even under dynamic UI conditions.

The system is organized into modular components, each adhering to a single-responsibility design principle. These include: (1) the gaze-driven pointer control, which enables real-time cursor positioning based on eye movement; (2) the NLP-based command parsing module, which interprets speech input captured by Vosk and maps it to specific UI actions among clicks, text entry, and scrolling; and (3) the record-and-replay test generation module, which logs user interactions in Gherkin syntax and compiles them into executable test scripts compatible with WebdriverIO for automated testing.

\section{EyeNav Capabilities}
\subsection{Voice Commands}

The current implementation supports four voice commands: \textit{Click}, \textit{Input}, \textit{Go (up/down)}, and \textit{Navigate (back/forward)}. These commands were selected to reflect fundamental browser interactions typically performed via mouse and keyboard, providing a minimal yet functional command set suitable for a prototype. Each command corresponds to a distinct action: Click triggers a mouse click, Input captures and types dictated speech, Go scrolls the page vertically in the specified direction, and Navigate back or forward transitions between last and next pages in the browser history.

\subsection{NLP in multiple languages}
Accessibility also encompasses internationalization, and the system is designed to adapt to the user's preferred browser language. Currently, it supports both English and Spanish, with additional languages easily integrable by downloading the corresponding voice recognition model and translating the interface localization files.

\subsection{Test Case generation}

Interaction logging follows a structured hierarchy for element identification. For click events, the Chrome extension attaches a global event listener to the entire page, maintained dynamically via a \verb|MutationObserver| to accommodate changes in the DOM. Only interactions with semantically clickable elements, including hyperlinks, buttons, and similar controls, are considered. When such an element is clicked, the system attempts to identify it using a prioritized attribute hierarchy: \verb|href|, \verb|id|, \verb|className|, and, if necessary, an automatically computed \verb|xPath|. This hierarchical strategy ensures that the most stable and descriptive selector available is used for referencing the element in the generated test script.


\begin{lstlisting}
Feature: Replay of session on MM DD at HH:MM:SS [AM/PM]

@user1 @web
Scenario: User interacts with the web page named "Amazon.com. Spend less. Smile more."

    Given I navigate to page "https://www.amazon.com/"
    And I click on tag with id "twotabsearchtextbox"
    And I input "nike black shoes"
    And I click on tag with id "nav-search-submit-button"
    And I scroll down
    And I click on tag with xpath "/html[1]/body[1]/div[1]/div[1]/div[1]/div[1]/div[1]/span[1]/div[1]/div[9]/div[1]/div[1]/span[1]/div[1]/div[1]/div[1]/span[1]/a[1]/div[1]"
\end{lstlisting}

The latter is an example of a generated test file. Each instruction is written in Gherkin syntax, promoting both human readability and maintainability. This format is particularly accessible to non-technical stakeholders while remaining fully compatible with the automated testing frameworks employed in the system.

\subsection{Accessible Interaction}
EyeNav supports keyboard navigation via the Tab key, enabling users to traverse interactive elements.

\section{EyeNav Use Cases}

Since each component operates independently, this flexibility allows the system to support a variety of use cases, each leveraging different capabilities of the tool.

\subsection{Accessible Interaction Mechanism for Web Applications}
Users with motor impairments or those seeking hands-free control can benefit from the gaze-based and voice-driven interface, offering an accessible and intuitive method of web navigation without the need for traditional input devices.

\subsection{Record-and-Replay Testing Tool (A-TDD)}

The logger module functions independently of the input modality, meaning it can be used even with conventional mouse and keyboard input. As such, EyeNav also serves as a lightweight, behavior-driven development (BDD) tool suitable for acceptance test-driven development (A-TDD) workflows.

\subsection{Support for Accessibility Evaluation Professionals}

The system provides a valuable platform for accessibility consultants, QA engineers, and researchers conducting accessibility evaluations. Its support for multimodal input—via eye tracking, voice recognition, or a combination of both—enables flexible testing setups tailored to a wide range of users and scenarios, including those that simulate real-world accessibility constraints.

\subsection{Intelligent Agents and Automation Scenarios}

EyeNav's architecture allows for the integration of intelligent agents capable of interpreting and executing multimodal input. In envisioned scenarios, agents powered by NLP models could use EyeNav to autonomously interact, also using speech or other simulated modalities, with web interfaces. These interactions can also be recorded using the logger module, enabling automated usability assessments and supporting research in autonomous accessibility testing.

\section{Results}

\subsection{User Feedback}

Qualitative interviews identified several key usability factors. Larger UI elements significantly improved gaze accuracy, making it easier for users to select targets with their eyes. Voice commands were most effective when they were short and distinct, reducing recognition errors. Environmental noise was found to negatively impact the reliability of speech recognition, suggesting the need for robust filtering or alternative input strategies. Users also indicated that visual indicators for "gaze hover targets" would enhance feedback and confidence during interaction. Additionally, simplified scroll commands were perceived as more usable and intuitive compared to earlier, more granular versions.

\subsection{Accessibility Insights}

Testers noted that this input method offers clear benefits for users with limited motor function. However, improvements in UI design (e.g., reachable icons) are needed for full accessibility compliance. Due to scope limitations, no participants with motor impairments were included, though future studies aim to address this.

\section{Discussion}

The integration of eye-tracking and NLP proved effective for hands-free interaction in a browser context. Further work is needed to improve performance in varied environments (e.g., low-light, users with glasses, non-native accents). Eye-based clicking was responsive but may require calibration for precision.

The testing module provided reproducible, human-readable test cases for interaction flows. While brittle on dynamic pages, these cases proved valuable for visual regression or task analysis. 

This architecture design allows for the combination of many of the purposes eye-tracking has already; we can analyze user behavior, identify usability bottlenecks, and validate system performance under varying conditions. This is especially valuable in eye-tracking based systems, where subtle differences in gaze patterns can significantly influence interaction outcomes. Replay tools also facilitate comparative evaluations, allowing different interface versions or input modalities to be tested using identical interaction sessions.



% !TEX root = main.tex

\section{Conclusion \& Future Work}
%    - Summarize main contributions: novel tool, accessibility impact, testing automation.
%    - Emphasize readiness for live demo and replicability (public code/datasets).
% TODO
a



% \section*{Acknowledgment} 


\balance
\bibliographystyle{IEEEtran}
\bibliography{local,bib/testing,bib/tools}

\end{document}
